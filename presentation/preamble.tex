\usepackage[hyperref,backend=biber,
% Exemples de styles: alphabetic, ieee, nature, numeric, verbose-trad1 (en utilisant \footcite{}).
% https://www.overleaf.com/learn/latex/Biblatex_bibliography_styles
style=alphabetic,
% backref=true,
% backrefstyle=three,
]{biblatex}

\newif\ifwebcast
\webcastfalse
\newcommand{\webcast}{\webcasttrue}
% \webcast
% \newcommand<>{\script}[1]{\note#2{{#1}}}
% \newcommand<>{\script}[1]{\note{\only#2{#1}}}
\def\script{\note}

\definecolor{supelecRed}{RGB}{120,30,56}
\definecolor{supelecPurple}{RGB}{104,95,115}
\usetheme{Thesis}

\usepackage[utf8]{inputenc}
\usepackage[english]{babel}
\usepackage{tikz}
\usepackage{bm}
\usepackage{ulem}
\usepackage{parcolumns}
\usepackage{multicol}
\usepackage{booktabs}
\usepackage[french]{isodate}
\usepackage[ruled,noend,algo2e]{algorithm2e}
\usepackage[T1]{fontenc}
\usepackage{lmodern} % Assurer une bonne impression!
\usepackage{tikz} % tikz est utilise pour tracer des boites, par exemple
\usepackage{pgfplots}
\usepackage{fix-cm}
\usepackage{grffile}
\usepackage{pgfpages}
\usepackage{xparse}

\usepackage{pifont} % Pour utiliser des symboles divers.
\usepackage{color}
\usepackage{comment}
\usepackage{xargs}
\usepackage[author={Accacio}]{pdfcomment}

\usepackage{mathtools}
\usepackage{amsmath}
\usepackage{amsthm}
\usepackage{mathrsfs}
\usepackage{mathbbol}

\usepackage{eucal}

\usepackage{subcaption}
\usepackage{caption}
\captionsetup{justification=centering}
\usepackage{float}
\usepackage{array}

\usepackage{xr}
\usepackage{subfiles}
\usepackage[math]{blindtext}
\usepackage{ifthen} % Entrer valeurs bool\'{e}ennes et autres options
\usepackage{csquotes} % Assurer les guillemets français

\usepackage{nicefrac,xfrac}
\usepackage{etoolbox}
\usepackage{fontawesome}
\usepackage{mathabx}

\usepackage{animate}
\usepackage{colortbl}

\definecolor{mpc_agent}{RGB}{243, 146, 0} % logo necsys
\colorlet{mpc_agent}{supelecRed!90}
\definecolor{mpc_coordinator}{RGB}{235, 235, 235}
\colorlet{mpc_coordinator}{supelecRed!10}
\definecolor{mpc_green}{RGB}{98, 160, 98}
\newcommand\encircle[1]{%
  {
  \usebeamerfont*{item projected}%
  \usebeamercolor[bg]{item projected}%
  \begin{pgfpicture}{-1ex}{0ex}{1ex}{2ex}
    \pgfpathcircle{\pgfpoint{0pt}{.75ex}}{1.4ex}
    \pgfusepath{fill}
    \pgftext[base]{\color{fg}#1}
  \end{pgfpicture}%
  }
}

\newcommand{\tikzmark}[1]{\tikz[baseline={(#1.base)},overlay,remember picture] \node[outer sep=0pt, inner sep=0pt] (#1) {\phantom{A}};}

\newcommand{\booksymbol}{\lower4pt\hbox{\pgfuseimage{beamericonbook}}}
\tikzset{%
  show controls/.style={
    postaction={
      decoration={
        show path construction,
        curveto code={
          \draw [blue]
          (\tikzinputsegmentfirst) -- (\tikzinputsegmentsupporta)
          (\tikzinputsegmentlast) -- (\tikzinputsegmentsupportb);
          \fill [red, opacity=0.5]
          (\tikzinputsegmentsupporta) circle [radius=.2ex]
          (\tikzinputsegmentsupportb) circle [radius=.2ex];
        }
      },
      decorate
    }}}

\usetikzlibrary{graphs,quotes,graphs.standard}
\usetikzlibrary{plotmarks}
\usetikzlibrary{arrows.meta}
\usepgfplotslibrary{patchplots}
\usetikzlibrary{calc,shapes,positioning}
\usetikzlibrary{math}
\usetikzlibrary{overlay-beamer-styles}
\tikzset{
  graphs/nodes={draw,circle,inner sep=1pt},
  <->/.style={latex-latex},
  ->/.style={-latex},
  <-/.style={latex-},
}
\usetikzlibrary {chains}
\usetikzlibrary{arrows.meta}
\usetikzlibrary{3d}
\usetikzlibrary{perspective}
\usetikzlibrary{calc,shapes,positioning,intersections}
\usetikzlibrary{overlay-beamer-styles}
 \usetikzlibrary{plotmarks}
  \usetikzlibrary{arrows.meta}
\usetikzlibrary{babel} % to correct problem with babel

% \usepackage{times}
% Or whatever. Note that the encoding and the font should match. If T1
% does not look nice, try deleting the line with the fontenc.
\graphicspath{{../img/}}

\usepackage{amssymb}
\usepackage{accents}

\SetKwRepeat{Do}{do}{while}


\newcommand{\eq}[2]{\mbox{$#1=#2$}}
\newcommand{\N}{\mathbb{N}}
\newcommand{\Z}{\mathbb{Z}}
\newcommand{\Q}{\mathbb{Q}}
\newcommand{\R}{\mathbb{R}}
\newcommand{\C}{\mathbb{C}}
\newcommand{\Np}{N_{\text{p}}}
\newcommand{\T}{^{\mathrm{T}}}
\newcommand{\1}{\mathbf{1}}
\newcommand{\0}{\mathbf{0}}
\newcommand{\abs}[1]{\left\lvert#1\right\rvert}
\newcommand{\norm}[1]{\left\lVert#1\right\rVert}
\newcommand{\Varepsilon}{\mathcal{E}}
\newcommand{\diff}{\mathop{}\mathopen{}\mathrm{d}}
\newcommand{\set}[1]{\mathcal{#1}}
\newcommand{\graph}[1]{\mathscr{#1}}
\newcommand{\p}{^{(p)}}
\newcommand{\pplusone}{^{(p+1)}}
\renewcommand{\vec}[1]{\boldsymbol{#1}}
\newcommand{\setbuild}[2]{\{#1\mid#2\}}
\newcommand{\seq}[2][n]{\lbrace #2_{0},\ldots,\,#2_{#1} \rbrace}
\newcommand{\hadamard}[2]{#1\circ #2}
\newcommand{\kron}[2]{#1\otimes#2}
\newcommand{\symmetric}{\mathbb{S}}
\newcommand{\semidefpos}{\mathbb{S}_{+}}
\newcommand{\defpos}{\mathbb{S}_{++}}
\newcommand{\elem}[2][1]{{#2}_{(#1)}}
\renewcommand{\implies}{\Rightarrow}
\renewcommand{\iff}{\Leftrightarrow}

\DeclareMathOperator{\fix}{fix}
\DeclareMathOperator{\Proj}{Proj}
\DeclareMathOperator{\dom}{dom}
\DeclareMathOperator{\card}{\#}
\DeclareMathOperator{\vectorize}{vector}
% \DeclareMathOperator{\vector}{vec}
%

% Theorem
% \newtheorem{thm}{Theorem}[section]
% \newtheorem{lem}[thm]{Lemma}

\newcommand{\nsubsystems}{M}
\newcommand{\umax}{\vec{u}_{\mathrm{\max}}}
\newcommand{\predhorz}{N_{p}}

\NewDocumentCommand \mpcvec { s m o o o } {%
  \IfBooleanTF{#1}{
    \def\optim{^\star}
  }{
    \def\optim{}
  }
  \IfValueTF{#5}{
    \vec{#2}_{#3}\optim{}[#4|#5]
  }{
    \IfValueTF{#4}{
      \vec{#2}_{#3}\optim{}[#4]
    }
    {
    \IfValueTF{#3}{
      \vec{#2}_i\optim{}[#3]
    }
    {
      \vec{#2}_i\optim{}[k]
    }
    }
  }
}


\NewDocumentCommand \mpcval { s m o o o } {%
  \IfBooleanTF{#1}{
    \def\optim{^\star}
  }{
    \def\optim{}
  }
  \IfValueTF{#5}{
    {#2}_{#3}\optim{}[#4|#5]
  }{
    \IfValueTF{#4}{
      {#2}_{#3}\optim{}[#4]
    }
    {
    \IfValueTF{#3}{
      {#2}_i\optim{}[#3]
    }
    {
      {#2}_i\optim{}[k]
    }
    }
  }
}






\newcommand{\uikk}{\mpcvec{u}[i][k][k]}
\newcommand{\optuikk}{\mpcvec*{u}[i][k][k]}

\newcommand{\globobj}{\mpcval{J}[][k]}
\newcommand{\optglobobj}{\mpcval*{J}[][k]}

\newcommand{\obji}{\mpcval{J}[i][k]}
\newcommand{\optobji}{\mpcval*{J}[i][k]}

\newcommand{\xik}{\mpcvec{x}}
\newcommand{\fik}{\mpcvec{f}}
\newcommand{\uik}{\mpcvec{u}}
\newcommand{\uiseq}{\mpcvec{u}[i][k:k+\predhorz-1][k]}
\newcommand{\optuiseq}{\mpcvec*{u}[i][k:k+\predhorz-1][k]}

\newcommand{\useq}{\mpcvec{u}[ ][k:k+\predhorz-1][k]}
\newcommand{\optuseq}{\mpcvec*{u}[ ][k:k+\predhorz-1][k]}
\newcommand{\Uik}{\mpcvec{U}}
\newcommand{\optUik}{\mpcvec*{U}}
\newcommand{\optuncUik}{\mpcvec*{\mathring{U}}}
\newcommand{\optuncU}{{\mathring{\vec{U}}^{\star}}}

\newcommand{\vik}{\mpcvec{v}}
\newcommand{\wik}{\mpcvec{w}}
\newcommand{\wiseq}{\mpcvec{w}[i][k:k+\predhorz-1][k]}
\newcommand{\Wik}{\mpcvec{W}}

\newcommand{\qik}{\mpcvec{q}}
\newcommand{\qiseq}{\mpcvec{q}[i][k:k+\predhorz-1][k]}
\newcommand{\thetaik}{\mpcvec{\theta}}
\newcommand{\optthetaiseq}{\mpcvec*{\theta}[i][k:k+\predhorz-1][k]}
\newcommand{\thetaseq}{\mpcvec{\theta}[][k:k+\predhorz-1][k]}
\newcommand{\optthetaseq}{\mpcvec*{\theta}[][k:k+\predhorz-1][k]}
\newcommand{\thetai}{\vec{\theta}_i}
\newcommand{\optthetai}{\vec{\theta}_i^{\star}}

\newcommand{\dik}{\mpcvec{d}}
\newcommand{\diseq}{\mpcvec{d}[i][k:k+\predhorz-1][k]}
\newcommand{\lambdaik}{\mpcvec{\lambda}}
\newcommand{\lambdai}{\vec{\lambda}_i}
\newcommand{\lambdaicheat}{\tilde{\vec{\lambda}}_i}
\newcommand{\optlambdai}{\vec{\lambda}_i^{\star}}

\newcommand{\Tik}{\mpcval{T}}



\newtheorem{assumption}{Assumption}%[numberby]
\newtheorem{assumptions}[assumption]{Assumptions}
\newtheorem{remark}{Remark}

\newif\ifdebug%
\newcommand{\draft}{\debugtrue}
\newcommand{\final}{\debugfalse}
\newcommand{\todo}[2][FORGOT TO DO SOMETHING]{\ifdebug%
  {%
    \color{red}
    #2}\else \PackageError{}{#1}{#2}#2\fi}%
\newcommand\doing[2][FORGOT TO DO SOMETHING]{\ifdebug%
  {%
    \color{blue}
    #2}\else \PackageError{}{#1}{#2}#2\fi}%
\newcommand\warning[1]{\ifdebug%
  {%
    \color{red}
    #1}\fi}
