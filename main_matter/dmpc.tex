\documentclass[../main.tex]{subfiles}

\begin{document}
\part{Distributed Model Predictive Control}
\chapter{Model Predictive Control}
\epigraph{\centering There are few people, however, who, if you told them a result, would be able to evolve from their own inner consciousness what the steps were which led up to that result.}
{\textit{A Study in Scarlet}\\ \textsc{Sir Arthur Conan Doyle}}
\minitoc


\begin{description}
  \item[System]
  \item[Prediction]
  \item[Optimization Problems to solve]
  \item[Applying control and Recursive Horizon Strategy]
\end{description}

\printbibliography%


\chapter{Decomposing the Model Predictive Control}
\epigraph{\centering The mystery of \\ the universe \\ is not time\\ but size.}
{\textit{The Gunslinger}\\\textsc{Stephen King}}

\minitoc

\begin{description}
  \item[Problem with size] the problem with solving a centralized MPC is blabla big blabla computationally expensive
  \item[State of the art] \cite{ChristofidesEtAl2013,ArauzEtAl2021,NotarnicolaNotarstefano2020} as base for discussion,
        present the nomenclatures for dMPC frameworks, centralization and so not, and finally a summary table showing where we want to focus:
        \begin{itemize}
          \item linear discrete-time;
          \item constraint-coupled;
          \item decomposable;
          \item non-cooperative;
          \item decentralized;
        \end{itemize}

\end{description}

\begin{remark}
  I would like to propose a change in nomenclature: usually decentralized control means agents do not communicate, so I think it should be more adequate to call it uncoordinated control, since there is no consensus between agents nor coordinator to referee. Then centralized or decentralized have something to do with topology.
\end{remark}

\printbibliography%

\chapter{Primary Decomposition-based distributed Model Predictive Control}
\begin{description}
  \item[decomposition]
  \item[negotiation]
  \item[Interpretation of variables]
  \item[simple example to illustrate convergence]
\end{description}


\printbibliography%


\end{document}
