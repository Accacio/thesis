\documentclass[../main.tex]{subfiles}

\begin{document}

\chapter[Resilient Primal Decomposition-based dMPC for scarce systems]{Resilient \\Primal Decomposition-based \\distributed \\Model Predictive Control\\ for scarce systems}\label{sec:safe_pddmpc_eq}
\epigraph{\centering So you tell me 'trust me' \\I can trust you\\ Just let me show you \\But I gotta work it out in a shadow of doubt \\'Cause I don't know if I know you}
{\textit{Lie}\\\textsc{Kevin Moore}}

In this chapter, we focus on the analysis of the primal decomposition-based \dmpc{} problem for scarce systems when in the presence of the attack chosen in~\S\ref{sec:anomalous}.

With the knowledge acquired after the analysis, we propose a method to detect the attacks and mitigate its effects.

\minitoc


\section{Scarce systems under attack}\label{sec:analys-scarce-syst}

Different from~\cite{VelardeEtAl2018,MaestreEtAl2021} which propose robust solutions, we propose a resilient \dmpc{} based on a hybrid of analytical/learning active detection method.
For this end, we begin with an analysis of the system under attack.

\subsection{Scarce Systems}\label{sec:scarce-systems}
First we recall the monolithic \mpc{} equivalent problem~\eqref{eq:qp_standard_form}, once more reproduced for the reader's convenience:
\begin{equation}
  \label{eq:qp_standard_form_again}
  \tag{\ref*{eq:qp_standard_form}}
  \begin{aligned}
    \begin{matrix}
      \underset{\vec{U}[k]}{\mathrm{minimize}} &
                                                 \frac{1}{2}\norm{\vec{U}[k]}^{2}_{H} + {\vec{f}[k]}^{T}\vec{U}[k] &\\
      \mathrm{subject~ to} &
                             \bar{\Gamma}\vec{U}[k]\preceq {\vec{U}}_{\text{max}}
    \end{matrix}
  \end{aligned},
\end{equation}
and the local problems~\eqref{eq:DOP_local} solved in the primal decomposition, also reproduced,

\begin{equation}
  \label{eq:DOP_local_reprise}
  \tag{\ref*{eq:DOP_local}}
  \bar{J}_{i}^{\star}(\thetaik)=
  \begin{matrix}
    \underset{\vec{U}_{i}[k]}{\mathrm{minimize}}&\obji=\frac{1}{2}\norm{\vec{U}_{i}[k]}^{2}_{H_{i}} + {\vec{f}_{i}[k]}^{T}\vec{U}_{i}[k]\\
    \mathrm{subject~ to} & \bar{\Gamma}_{i}\vec{U}_{i}[k] \preceq \thetaik:\lambdaik
  \end{matrix}.
\end{equation}

The unconstrained version of problems~\eqref{eq:qp_standard_form_again} and~\eqref{eq:DOP_local_reprise},
\begin{align}
  \label{eq:qp_standard_form_unconstrained}
  \begin{aligned}
    \begin{matrix}
      \underset{\vec{U}[k]}{\mathrm{minimize}} &
                                                 \frac{1}{2}\norm{\vec{U}[k]}^{2}_{H} + {\vec{f}[k]}^{T}\vec{U}[k] &\\
    \end{matrix},
  \end{aligned}\\
  \label{eq:DOP_local_unconstrained}
  \begin{aligned}
    \begin{matrix}
    \underset{\vec{U}_{i}[k]}{\mathrm{minimize}}&\frac{1}{2}\norm{\vec{U}_{i}[k]}^{2}_{H_{i}} + {\vec{f}_{i}[k]}^{T}\vec{U}_{i}[k]\\
    \end{matrix}
  \end{aligned}
\end{align}
have analytical solutions~\cite{BoydVandenberghe2004}
\begin{align}
  \label{eq:qp_standard_form_unconstrained_solution}
  \vec{U}_{\text{unc}}^{\star}[k]=-H^{-1}\vec{f}[k],\\
  \label{eq:DOP_local_unconstrained_solution}
  \vec{U}_{i_{\text{unc}}}^{\star}[k]=-H_{i}^{-1}\vec{f}_{i}[k].
\end{align}

We call a system scarce when its unconstrained solution $\vec{U}_{\text{unc}}^{\star}[k]$ lies outside the bounds of the polytope formed by the constraints, i.e.,
\begin{equation}
\bar{\Gamma}\vec{U}_{\text{unc}}[k]\succ {\vec{U}}_{\text{max}}.
\end{equation}

Furthermore, we assume that for all sub-systems, their unconstrained solutions $\vec{U}_{i_{\text{unc}}}^{\star}[k]$ neither respect the local constraints
\begin{equation}
\bar{\Gamma}_{i}\vec{U}_{i_{\text{unc}}}^{\star}[k]\succ {\vec{U}}_{\text{max}}, \forall i\in\set{M}.
\end{equation}

This means that the optimal solution would need more resources (more than $\vec{U}_{\max}$). So the solution of those \qp{} result on the projection of the solutions onto the polytope, which results being projected on the perimter of the polytope.

We assume that given projection solves the inequality constrained problem with same solution for an equality constraint problem
\begin{equation}
  \label{eq:DOP_local_equality}
  \tag{\ref*{eq:DOP_local}}
  \bar{J}_{i}^{\star}(\thetaik)=
  \begin{matrix}
    \underset{\vec{U}_{i}[k]}{\mathrm{minimize}}&\obji=\frac{1}{2}\norm{\vec{U}_{i}[k]}^{2}_{H_{i}} + {\vec{f}_{i}[k]}^{T}\vec{U}_{i}[k]\\
    \mathrm{subject~ to} & \bar{\Gamma}_{i}\vec{U}_{i}[k] = \thetaik:\lambdaik
  \end{matrix}.
\end{equation}

\begin{remark}
  Observe that this is not always true using the same $\bar{\Gamma}_{i}$ for both problems.
  The constraints in~\eqref{eq:qp_standard_form_again} and~\eqref{eq:DOP_local_reprise}, can be interpreted as the intersection of the halfspaces described by the lines of $\bar{\Gamma}$ and $\vec{U}_{\max}$, i.e., the halfspaces ${\set{S}_i=\setbuild{\vec{x}}{\elem[i,\star]{\Gamma}\vec{x}\leq\elem[i,\star]{\vec{U}_{\max}}}}$, and the intersection ${\set{S}=\setbuild{\vec{x}}{\Gamma\vec{x}\preceq\vec{U}}=\bigcap\limits_{i=1}^{\card\vec{U}_{\max}}}\set{S}_{i}$.
  For this to be true, only the halfspaces closest to the unconstrained solution are kept and transformed into intersection of hyperplanes (equality constraint).
  This computation may be costly depending on the number of halfspaces used.
  For our study cases where usually the values of $\Gamma$ are positive, we assume we can find the corresponding problem.
\end{remark}

\subsection{Why scarce systems?}\label{sec:why-scarce-systems}


are forced to compromise
This~\eqref{eq:qp_standard_form_again}

\begin{enumerate}
  \item Analysis of eigenvalues of negotiation
\end{enumerate}
\begin{equation}
  \begin{matrix}
    \lambdaik=\underset{\lambdaik}{\mathrm{argmax}}\{\underset{\vec{U}_i[k]}{\mathrm{minimize}}\;\frac{1}{2}U_i^TH_iU_i+F_i^TU_i+c_i+\lambda_i^T(U_i-\theta_i^{(p)})\}\\
  \end{matrix}
\end{equation}

\begin{figure}[h]
  \centering
  \begin{subfigure}{.45\textwidth}
    \includegraphics[width=\textwidth]{../img/original-minimum.png}
    \caption{Original minimum.}
    \label{fig:first}
  \end{subfigure}
  \hfill
  \begin{subfigure}{0.45\textwidth}
    \includegraphics[width=\textwidth]{../img/new-minimum-selfish.png}
    \caption{Minimum after attack.}
    \label{fig:second}
  \end{subfigure}
  \caption{Effects of non-conforming behaviors on optimal value. \todo{Refaire les images}}
  \label{fig:figures}
\end{figure}

\begin{figure}[h]
  \centering
  \begin{subfigure}{0.45\textwidth}
    \includegraphics[width=\textwidth]{../img/ignoreX.png}
    \caption{Optimal value after ignoring attacker.}
    \label{fig:third}
  \end{subfigure}
  \hfill
  \begin{subfigure}{0.45\textwidth}
    \includegraphics[width=\textwidth]{../img/correctX.png}
    \caption{Optimal value after trying to recover original behavior.}
    \label{fig:third}
  \end{subfigure}

  \caption{Recovery options.}\label{fig:figures}
\end{figure}

\section{Study case}
\label{sec:study-case}

Model 3R-2C \cite{GoudaEtAl2002}
\begin{figure}[h]
  \centering
  \begin{circuitikz}[european]
    \draw (0,0) node[tlground]{} to[isource, l=$P^{\text{heat}}$] ++(0,2) to[short, -*] ++(1.5,0) coordinate (a);

    \draw (a) node[above]{$T^{\text{in}}$}  to[C=$C^{\text{air}}$] ++(0,-2) node[tlground]{};
    \draw (0,-3) node[tlground]{} to[isource, l=$I^{\text{sol}}$] ++(0,2)
    to[short, -*] ++(1.5,0) coordinate (b);
    \draw (b) to[C=$C^{\text{walls}}$] ++(0,-2) node[tlground]{};

    \draw (a) -- ++(2,0) coordinate (c) -- ++(0,-.5) to[R=$R^{\text{iw/ia}}$] ++(0,-2) -- ++(0,-.5) coordinate (d);

    \draw (b) node[above]{$T^{\text{walls}}$} to[short,-*] (d);

    \draw (c) --  ++(2.5,0) -- ++(0,-.5) to[R=$R^{\text{oa/ia}}$] ++(0,-2) -- ++(0,-.5) coordinate (e);

    \draw (d) to[R=$R^{\text{ow/oa}}$] (e) to[battery,l=$T^{\text{out}}$] ++(0,-2) node[tlground]{};
  \end{circuitikz}
  \caption{Thermic Model 3R-2C of a room.}
  \label{fig:3R2C_model}
\end{figure}

The state-space model of each subsystem is given by:
\begin{equation}
  \begin{matrix}
    \label{eq:systems_cont}
    \dot{\vec{x}}_{i}(t)  &=&{A_{c}}_{i}\vec{x}_{i}(t) &+& {B_{c}}_{i}\vec{u}_{i}(t)\\
    \vec{y}_{i}(t)        &=&{C_{c}}_{i}\vec{x}_{i}(t) &&
  \end{matrix},
\end{equation}
where
\begin{equation}
  \label{eq:4}
  \begin{matrix}
    A_{\mathrm{c}_{i}}=\left[
    \begin{matrix}
      -\frac{1}{C^{\text{walls}}_{i}R^{\text{oa/ia}}_{i}}-\frac{1}{C^{\text{walls}}_{i}R^{\text{iw/ia}}_{i}}& \frac{1}{C^{\text{walls}}_{i}R^{\text{iw/ia}}_{i}}\\
      \frac{1}{C^{\text{air}}_{i}R^{\text{iw/ia}}_{i}} &-\frac{1}{C^{\text{air}}_{i}R^{\text{ow/oa}}o_{i}}-\frac{1}{C^{\text{air}}_{i}R^{\text{iw/ia}}_{i}}
    \end{matrix}\right]\\
    \begin{matrix}
      B_{\mathrm{c}_{i}}=\left[
      \begin{matrix}  \frac{10}{C^{\text{walls}}_{i}}& 0\end{matrix}
                                                       \right]\T&C_{\mathrm{c}_{i}}=\left[\begin{matrix}1 & 0\end{matrix}\right]
    \end{matrix}
  \end{matrix}
\end{equation}
where ${\vec{x}_{i}=[{{x}_{A}}_{i}\T\ {{x}_{W}}_{i}\T]\T}$. ${x_A}_i$ and ${x_W}_i$ are the mean temperatures of the air and walls inside room~$i$. $\vec{u}_{i}$ is the input (the heating power)
for the corresponding room. The inputs are constraint by ${\sum_{i=1}^{4}\vec{u}_{i}(t)\preceq 4\mathrm{kW}}$.

\begin{table}[b]
  \centering
  \caption{Model Parameters}\label{tab:modelParamMeaning}
  \begin{tabular}[b]{cl}
    \toprule
    Symbol&Meaning\\
    \midrule
    $C^{\text{air}}_{i}$  &Heat Capacity of Inside Air\\
    $C^{\text{walls}}_{i}$ &Heat Capacity of External Walls\\
    $R^{\text{iw/ia}}_{i}$ &Resist. Between Inside Air and Inside Walls\\
    $R^{\text{ow/oa}}_{i}$ &Resist. Between Outside Air and Outside Walls\\
    $R^{\text{oa/ia}}_{i}$ &Resist. Between Inside and Out.\ Air (from windows)\\
    \bottomrule
  \end{tabular}
\end{table}

\begin{table}[b]
  \centering
  \caption{
    Parameters for each agent}\label{tab:modelParam}
  \begin{tabular}[t]{cccccc} \toprule
    Symbol& I & II & III & IV &Unit\\
    \midrule
    $C^{\text{walls}}$   &$5.4$&$4.9$&$4.7$&$4.7$ &$10^{4}\mathrm{J/K}$ \\
    $C^{\text{air}}$     &$7.5$ &$8.4 $&$8.2$ &$7.7$&$10^{4}\mathrm{J/K}$  \\
    $R^{\text{oa/ia}}$   &$5.2$&$4.6$&$4.9$&$5.4$&$10^{-3}\mathrm{K/W}$ \\
    $R^{\text{iw/ia}}$   &$2.3$&$2.4$&$2.3$&$2.9$&$10^{-4}\mathrm{K/W}$\\
    $R^{\text{ow/oa}}$   &$1.5$&$0.6$&$0.7$&$0.7$& $10^{-4}\mathrm{K/W}$ \\
    \bottomrule
  \end{tabular}
\end{table}

the attacker has no motivation to get more

\end{document}
