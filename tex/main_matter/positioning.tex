\documentclass[../main.tex]{subfiles}

\begin{document}

\chapter*{Choices \& Stance}\label{sec:positioning}
\epigraph{\centering There are few people, however, who, if you told them a result, would be able to evolve from their own inner consciousness what the steps were which led up to that result.}
{\textit{A Study in Scarlet}\\ \textsc{Sir Arthur Conan Doyle}}
As we have seen in the last chapters, there are many steps a designer has to take to design a \cps{} and decompose the \mpc{}.
Moreover, if we are interested in security, some more thinking must be put into it, and choices to be made.

In this brief chapter, we retrace the steps, recapitulating the choices made for this work mentioned in the course of the chapters.
The idea of this chapter is to concentrate them all in one place so a reader can have a better global vision and know more or less what to expect (or not) from the next chapters.

We divide it into two sections.
The first is dedicated to the choices regarding the design of the system's control.
The second one is for the choices for security.
In the latter, we compare this thesis with the most adjacent works, already presented in Chapter~\ref{sec:anomalous}.

\section*{Control design}
\paragraph*{Control}\label{sec:control_chosen}
We model the system as a \textbf{\ltidt{} system} with \textbf{linear input constraints}, and we choose to control it using \textbf{\mpc{}}.
The objective function used is \textbf{quadratic}, which by consequence produces a \mpc{} that solves a \textbf{\qp{} problem}.
However, we suppose the problem may impact the calculation if we try to solve it in a monolithic form. We need to \textbf{decompose} it.

\paragraph*{Decomposition}\label{sec:decomposition_chosen}
We choose to use the \textbf{optimization decomposition framework}, shown in~\cite{ConejoEtAl2006,BoydEtAl2015}.
We suppose the system is \textbf{decomposable} into \textbf{sub-systems} and our original problem is decomposable into \textbf{sub-problems}.

Using the framework, we see the input constraints \textbf{couple} the sub-systems, and as a consequence, the sub-problems.
The system is decomposed using the configuration for \textbf{complicating constraints}. Finally, we choose to use a decomposition called \textbf{primal decomposition}, detailed in the next chapter, which needs \textbf{coordination} to find the optimal solution.

\paragraph*{Topology}\label{sec:decomposition_chosen}
The first choice for the system's topology was to assume \textbf{correspondence} between sub-systems and sub-problems.

The agents were supposed to form a \textbf{connected} graph, in which the agents can communicate to coordinate the decomposition (\textbf{coordinated control}).
For some security/trust reasons we choose to use an \textbf{hierarchical} topology, where we \emph{entrust} an agent the role of \textbf{referee}, coordinating the exchanges between other agents.

We suppose the communication is \textbf{bidirectional}, where the agents can communicate with the \textbf{referee} and vice-versa.
We also suppose the agents communicate \textbf{synchronously}, so we do not have delays in the communication.

\section*{Security}
For security, we use~\cite{VelardeEtAl2017b,ChanfreutEtAl2018} as inspiration to demonstrate the \textbf{vulnerabilities of the primal decomposition} method, which were never presented.

As the most adjacent works, we are only interested in \textbf{inter-agent attacks} of the \textbf{\fdi{}} type, whose exact model is presented in the next chapter.

In~\cite{VelardeEtAl2017b,VelardeEtAl2018}, the authors present robust methods based on graph theory, where the attacker is not detected per se, but suspect agents are disconnected or ignored somehow.
In~\cite{VelardeEtAl2017a,MaestreEtAl2021}, another robust method is used, this time employing a scenario-based approach, where suspect sequences receive different weights.

We opt, however, as in~\cite{AnandutaEtAl2018,AnandutaEtAl2019,AnandutaEtAl2020},
for an \textbf{active} \textbf{resilient} approach.
The strategy is precisely detailed in the next chapters, but we use an \textbf{active} method for the detection, where we add some data to detect the attacker.
We use a \textbf{hybrid} of \textbf{analytical knowledge} of the decomposition and \textbf{learning methods} to create the detector.
For the recovery, we use the same \textbf{analytical knowledge} used for the detection to reconstruct some variables and compensate for disturbances caused by the attack.

Due to the reduced number of adjacent works, we can compile and compare them in a tabular manner.
In Tab.~\ref{tab:compare_works}, we compare them with the methods presented in this thesis.

\begin{landscape}
  \vspace*{\fill}
\begin{table}[H]
  \centering
  \begin{tabular}[h]{lccccc}
    \toprule
     & Decomposition & Present vulnerabilities?  & Resilient/Robust & Detection & Mitigation\\
    \midrule
    \parbox{20pt}{\cite{VelardeEtAl2017a}\\ \cite{MaestreEtAl2021}} & Dual & Yes & Robust (Scenario) & NA & NA\\\\
    \parbox{20pt}{\cite{VelardeEtAl2017b} \\ \cite{VelardeEtAl2018}} & Dual & Yes & Robust (f-robust) & NA & NA\\\\
    \cite{ChanfreutEtAl2018} & Jacobi-Gauß & Yes & -- & -- & --\\\\
    \parbox{40pt}{\cite{AnandutaEtAl2018}\\\cite{AnandutaEtAl2019}\\\cite{AnandutaEtAl2020}} & Dual & Yes & Resilient& Analyt./Learn. & Disconnect (Robustness)\\\\
    Our & Primal & Yes & Resilient & Active Analyt./Learn. & Data reconstruction\\
    \bottomrule
  \end{tabular}
  \caption[Comparison between adjacent works]{Comparison between adjacent works. NA meaning not applicable.}\label{tab:compare_works}
\end{table}
  \vspace*{\fill}
\end{landscape}

\end{document}
